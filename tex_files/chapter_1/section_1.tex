\section{MOOC overview}
\label{sec:mooc_overview}

This section consists of an overview about a massive open online course (MOOC). A MOOC is an online course aimed at unlimited participation and open access via the web. In addition to traditional course materials such as filmed lectures, readings, and problem sets, many MOOCs provide interactive user forums to support community interactions between students, professors, and teaching assistants (TAs). MOOCs are a recent and widely researched development in distance education which were first introduced in 2008 and emerged as a popular mode of learning in 2012.[Pappano, Laura. "The Year of the MOOC". The New York Times. Retrieved 18 April 2014.][Lewin, Tamar (20 February 2013). "Universities Abroad Join Partnerships on the Web". New York Times. Retrieved 6 March 2013.]. Early MOOCs often emphasized open-access features, such as open licensing of content, structure and learning goals, to promote the reuse and remixing of resources. Some later MOOCs use closed licenses for their course materials while maintaining free access for students.  [Wiley, David. "The MOOC Misnomer". July 2012] [Cheverie, Joan. "MOOCs an Intellectual Property: Ownership and Use Rights". Retrieved 18 April 2013.] [David F Carr (20 August 2013). "Udacity hedges on open licensing for MOOCs". Information Week. Retrieved 21 August 2013.][P. Adamopoulos, "What Makes a Great MOOC? An Interdisciplinary Analysis of Student Retention in Online Courses", ICIS 2013 Proceedings (2013) pp. 1–21 in AIS Electronic Library (AISeL)].

The first MOOCs emerged from the open educational resources (OER) movement. The term MOOC was coined in 2008 by Dave Cormier of the University of Prince Edward Island in response to a course called Connectivism and Connective Knowledge (also known as CCK08). CCK08, which was led by George Siemens of Athabasca University and Stephen Downes of the National Research Council, consisted of 25 tuition-paying students in Extended Education at the University of Manitoba.
Alongside the development of these open courses, other E-learning platforms emerged - such as Khan Academy, Peer-to-Peer University (P2PU), Udemy and ALISON - which are viewed as similar to MOOCs and work outside the university system or emphasize individual self-paced lessons.[Yuan, Li, and Stephen Powell. MOOCs and Open Education: Implications for Higher Education White Paper. University of Bolton: CETIS, 2013. pp. 7–8.][ "What You Need to Know About MOOCs". Chronicle of Higher Education. Retrieved 14 March 2013.][ "Open Education for a global economy".][ Booker, Ellis (30 January 2013). "Early MOOC Takes A Different Path". Information Week. Retrieved 25 July 2013.][Bornstein, David (11 July 2012). "Open Education For A Global Economy". New York Times. Retrieved 25 July 2013.].
According to The New York Times, 2012 became "the year of the MOOC" as several well-financed providers, associated with top universities, emerged, including Coursera, Udacity
In January 2013, Udacity launched its first MOOCs-for-credit, in collaboration with San Jose State University. In May 2013 the company announced the first entirely MOOC-based master's degree, a collaboration between Udacity, AT\&T and the Georgia Institute of Technology, costing \$7,000, a fraction of its normal tuition.["Georgia Tech, Udacity Shock Higher Ed With \$7,000 Degree". Forbes. 2012-04-18. Retrieved 2013-05-30.]




