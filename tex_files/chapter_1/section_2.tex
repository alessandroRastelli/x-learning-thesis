\section{MOOC classification}
\label{sec:mooc_classification}

In a few years many MOOC services were developed. Each service has particular features and policies, and among these I have selected the most important and relevant that have inspired me with my project.

\subsection{Coursera}
\label{subsec:coursera}

\begin{figure}[htb] %  figure placement: here, top, bottom
 \centering
 \includegraphics[width=0.5\linewidth]{images/chapter1/coursera.jpg}\hfill
 \caption[Coursera logo]{Coursera logo}
 \label{fig:fourV}
\end{figure}

Coursera is a MOOC platform established by professors of Computer Science, Andrew Ng and Daphne Koller of Stanford University. Coursera is only a platform, each course depends on the universities that it provides. This platform has more than 150 International University Partners including the University of Rome La Sapienza and the Bocconi University of Milan that allow the platform to offer courses in many different languages, including Italian. All courses on Coursera are accessible for free.
For those interested, in some courses, they are given the opportunity to obtain a “Verified certificate” at the end of course, which is a certificate issued by the university providing the course. The certificates are paid and have a various costs.
Even Coursera courses generally begin at a certain date and have an average duration of a few weeks. They are divided into modules which are published online weekly. Forms can be made either by video contributions or by textual contributions. In the videos, furthermore the speed of the speech can be adjusted and in some cases they are provided with subtitles in many different languages.
For many courses intermediate quizzes are provided, generally relating to one group of modules at a time, and a final test of the course.
Each course has a discussion forum (divided into sections), a showcase for ad publication, a community that helps you get to know other students of the course and wiki, which is useful as external support to complete course units.

In less than a year, Coursera has received \$ 22 million in venture capital and has gained such interest as to induce certain American universities to wonder if it was worth it to get on this train.

\subsection{EdX}
\label{subsec:EdX}
\begin{figure}[htb] %  figure placement: here, top, bottom
 \centering
 \includegraphics[width=0.5\linewidth]{images/chapter1/edx_logo.png}\hfill
 \caption[EdX logo]{EdX logo}
 \label{fig:fourV}
\end{figure}

EDX is an initiative of online education non-profit organization founded in May 2012 by the Massachusetts Institute of Technology (MIT) and Harvard University,that have allocated \$ 30 milion to create EdX.
It offers free online courses and MOOC provided by MITX, HarvardX, BerkeleyX, UTX and many other universities, Edx differs from other MOOC providers, such as Coursera and Udacity, in that it is a nonprofit organization and runs on open-source software.[4][5]
More than 70 schools, nonprofit organizations, and corporations offer or plan to offer courses on the edX website.[6] 
As of 22 October 2014, edX has more than 4 million students taking more than 500 courses online.[7]

%%%%
Recently EDX in partnership with Google, has created the portal mooc.org, with the aim of becoming the largest container MOOC.
Often it is the same version of the courses given in the classroom, but with the addition of materials and online support that allow anyone to follow at a distance. At the end of the course, if you have passed all the tests, you can get a certificate, which differs from the original only in the final statement (MITX, HarvardX, BerkeleyX).

%%%%

EdX offers certificates of successful completion, but does not offer course credit. Whether or not a college or university offers credit for an online course is within the sole discretion of the school.["edX FAQs". edX. Retrieved August 26, 2012.]

Inolte offre la possibilità di ottenere certificati gratuiti non verificati,ma anche la possibilità di ottenere certificati d’onore verificati a pagamento.

EdX courses consist of weekly learning sequences. Each learning sequence is composed of short videos interspersed with interactive learning exercises, where students can immediately practice the concepts from the videos.[wikipedia]
Besides EdX offer online discussion forum where students can post and review questions and comments to each other and teaching assistants

EdX has been developed as open-source software and made available to other institutions of higher learning that want to make similar offerings. On June 1, 2013, edX open sourced its entire platform.[22].The source code can be found on GitHub.[9][23]


\subsection{SkillShare}
\label{subsec:SkillShare}
\begin{figure}[htb] %  figure placement: here, top, bottom
 \centering
 \includegraphics[width=0.5\linewidth]{images/chapter1/skillshare.jpg}\hfill
 \caption[SkillShare logo]{SkillShare logo}
 \label{fig:fourV}
\end{figure}


%http://www.forbes.com/sites/jjcolao/2012/08/07/learning-by-doing-skillshare-unveils-hybrid-classes/#2715e4857a0b48b2f14421c2


Skillshare, a New York based startup that runs an online marketplace for local classes.
Unlike traditional study, it introduces ‘Hybrid’ courses – month-long programs where students learn by collaborating on a guided projects rather than by absorbing material from a lecturer. Therefore, the SkillShare philosophy is focused on learning by doing.

As the name suggests, anyone with a skill of some kind is called to “teach” others by organizing virtual classes with users worldwide or promoting the courses held physically in an actual place.

Moreover, online classes at Skillshare are taught by industry experts. The courses, which are free, welcome anyone willing to learn. Most courses focus more on interaction than lecturing, and the goal of this is to learn by completing a project.
%[9][11]
In Skillshare all the online courses are self-paced and there are a few main categories of learning which are creative arts, design, entrepreneurship, lifestyle and technology.


Finally we can say that unlike other platforms, Skillshare allows to pay a monthly subscription of about \$10 or buy the single course.


\subsection{Udacity}
\label{subsec:udacity}
\begin{figure}[htb] %  figure placement: here, top, bottom
 \centering
 \includegraphics[width=0.5\linewidth]{images/chapter1/udacity.png}\hfill
 \caption[Udacity logo]{Udacity logo}
 \label{fig:fourV}
\end{figure}

Udacity was born in 2011 from an experiment at Stanford University and already collected 4 million members. The portal offers scientific, design and business courses which are divided by various levels. The courses are provided by non-traditional teaching methods.
The certificates are released at the end of the course if the candidate needs to prove his identity and has reached the objectives of the course without external aid. Some of Udacity's courses are fee.
Udacity’s courses are all self-paced they are organized primarily according modules. These modules have textual and video contributions, but also offer activities and short interviews with professors. In some cases there are provided English subtitles for the videos of which you can also adjust the speed of the speech.
Also the courses offered by Udacity have an ad showcase, forum, wiki and community. For most of the courses an intermediate quiz is provided, generally relating to a group of modules at a time, and a final test of the course. Unlike other platforms, on Udacity, quizzes can be repeated without it affecting the final grade.

A year ago, Udacity expanded its offering proposals to Nanodegree programs, which are a kind of diploma for those who want to improve and further develop their skills. Among the courses offered, there are front-end developers, data analysts and iOS developers. This opportunity offered by the platform has obviously attracted the attention of Google, which recently announced that it has launched in collaboration with the startup courses Tech Entrepreneur Nanodegree and Android Development.

With Udacity, each student learns by following online classes that last from 6 to 9 months. The cost is \$ 200 a month and whoever finishes on time receives 50\% of the tuition fees back.



\subsection{Udemy}
\label{subsec:udemy}
\begin{figure}[htb] %  figure placement: here, top, bottom
 \centering
 \includegraphics[width=0.5\linewidth]{images/chapter1/udemy.jpg}\hfill
 \caption[Udemy logo]{Udemy logo}
 \label{fig:fourV}
\end{figure}

Udemy is platform or marketplace for online learning that allows anyone to publish online video  courses. There is an Italian version of the site but the majority of the courses are in English and in other languages such as French, Spanish and Chinese.
There are free and paid courses in many different areas: excel, photoshop, facebook for companies, mathematical, but also cooking, music, sports, arts, languages.
It's an incredibly flexible platform especially for teachers, which automatically create their video courses, determine their cost and the “category”.
At the end of every course a certificate is released. Udemy provides a platform for experts of any kind to create courses which can be offered to the public, either at no charge or for a tuition fee, but in some cases it is possible to see some of the first lessons for free.
The Udemy courses are all self-paced. They are organized in modules but they are not released weekly. Most videos do not have subtitles. They have, moreover, all the most common tools of other platforms such as forums, communities, wikis and chat rooms; however, there is a great and original function for taking notes during lessons.
Udemy has made a special effort to attract corporate trainers seeking to create coursework for employees of their company.[Carr, David F. Udemy Comes To Corporate Training Information Week. April 16, 2013]