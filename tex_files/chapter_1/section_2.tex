\section{MOOC classification}
\label{sec:mooc_classification}

In few years many MOOC services are developed, each of these has the particular features, policies, different services and among these i have selected those which can be considered the most important or however of the most relevant and to which I was inspired.

\subsection{Coursera}
\label{subsec:coursera}

\begin{figure}[htb] %  figure placement: here, top, bottom
 \centering
 \includegraphics[width=0.5\linewidth]{images/chapter1/coursera.jpg}\hfill
 \caption[Coursera logo]{Coursera logo}
 \label{fig:fourV}
\end{figure}

Coursera is a platform MOOC Established by Professors of Computer Science Andrew Ng and Daphne Koller Stanford University.Specifically Coursera is only a platform, each course depends on the universities that it provides.This platform have more than 150 International  University Partners of Which anche the University of Rome La Sapienza and the Bocconi University of Milan, to allow the platform to offer courses in many different languages among which including Italian. All courses are accessible Coursera For free.
For those interested in some courses are given the opportunity to obtain, at the end of course, a “Verified certificate” a certificate issued by the university providing the course. The certificates are paid and have a variable cost.
Even Coursera courses generally begin at a precise date and have an average duration of a few weeks. They are divided into modules which are publishe online weekly. Forms can be made either by video contributions from both textual contributions. In the videos. Furthermore, the speed of the speech, can be adjusted and in some cases they are provided with subtitles in many different languages.
For most of the courses are provided for the quiz intermediate, generally relating to a group of modules at a time, and a final test of the course.
Each course has a discussion forums (divided into sections), a showcase for the publication of the ads, a community that helps you get to know other students of the course and wiki, useful as external support to complete course units. Unlike other platforms MOOC presented on this page, Coursera provides for the acceptance of an honor code before you can start your course.

%%%%%%%
In less than a year, Coursera has attracted \$ 22 million in venture capital and has aroused such interest as to induce certain American universities to wonder if it was worth it to get on this train
%%%%%%%

\subsection{EdX}
\label{subsec:EdX}
\begin{figure}[htb] %  figure placement: here, top, bottom
 \centering
 \includegraphics[width=0.5\linewidth]{images/chapter1/edx_logo.png}\hfill
 \caption[EdX logo]{EdX logo}
 \label{fig:fourV}
\end{figure}

EDX is an initiative of online education non-profit organization founded in May 2012 by the Massachusetts Institute of Technology (MIT) and Harvard University,that have allocated \$ 30 milion to create EdX.
It offers free online courses and MOOC provided by MITX, HarvardX, BerkeleyX, UTX and many other universities, Edx differs from other MOOC providers, such as Coursera and Udacity, in that it is a nonprofit organization and runs on open-source software.[4][5]
More than 70 schools, nonprofit organizations, and corporations offer or plan to offer courses on the edX website.[6] 
As of 22 October 2014, edX has more than 4 million students taking more than 500 courses online.[7]

%%%%
Often it is the same version of the courses given in the classroom, but with the addition of materials and online support that allow anyone to follow at a distance. At the end of the course, if you have passed all the tests, you can get a certificate, which differs from the original only in the final statement (MITX, HarvardX, BerkeleyX).

%%%%

EdX offers certificates of successful completion, but does not offer course credit. Whether or not a college or university offers credit for an online course is within the sole discretion of the school.["edX FAQs". edX. Retrieved August 26, 2012.]

EdX courses consist of weekly learning sequences. Each learning sequence is composed of short videos interspersed with interactive learning exercises, where students can immediately practice the concepts from the videos.
Besides EdX offer online discussion forum where students can post and review questions and comments to each other and teaching assistants

EdX has been developed as open-source software and made available to other institutions of higher learning that want to make similar offerings. On June 1, 2013, edX open sourced its entire platform.[22].The source code can be found on GitHub.[9][23]


\subsection{SkillShare}
\label{subsec:SkillShare}
\begin{figure}[htb] %  figure placement: here, top, bottom
 \centering
 \includegraphics[width=0.5\linewidth]{images/chapter1/skillshare.jpg}\hfill
 \caption[SkillShare logo]{SkillShare logo}
 \label{fig:fourV}
\end{figure}


%http://www.forbes.com/sites/jjcolao/2012/08/07/learning-by-doing-skillshare-unveils-hybrid-classes/#2715e4857a0b48b2f14421c2


Skillshare, a New York based startup that runs an online marketplace for local classes.
Unlike traditional study it’s introducing ‘Hybrid’ courses – month-long programs where students learn by collaborating on a guided project rather than by absorbing material from a lecturer, then the SkillShare philosophy are focus on learning by doing.
%%%%%
As the name suggests, anyone with a skill of some kind is called to “teach” others, by organizing virtual classes with users worldwide or promoting the courses held physically in a concrete place.
%%%%%
Moreover, online classes at Skillshare are taught by industry experts. The courses, which are not accredited, accept anyone who wants to learn.[10][11][12] The majority of courses focus more on interaction than lecturing, with the primary goal of learning by completing a project.[9][11] The main categories of learning are creative arts, design, entrepreneurship, lifestyle and technology, with subtopics covering a myriad of skills.[13][14][15][16]

Skillshare breaks down the categories of their classes into advertising, business, design, fashion and style, film and video, food and drink, music, photography, technology, and writing and publishing.[30][31] Within each category is a list of available courses being offered under that heading, which are taught by industry leaders.[31][32] All the online courses are self-paced.[2][33]

Finally we can say that unlike other platforms Skillshare by the ability to pay a monthly subscription of about \$10, or buy the single course


\subsection{Udacity}
\label{subsec:udacity}
\begin{figure}[htb] %  figure placement: here, top, bottom
 \centering
 \includegraphics[width=0.5\linewidth]{images/chapter1/udacity.png}\hfill
 \caption[Udacity logo]{Udacity logo}
 \label{fig:fourV}
\end{figure}

Udacity was born in 2011 from an experiment at Stanford University and already collected 4 million members. The portal offers scientific, of design and business courses, divided by the various levels. The courses are provided to non-traditional teaching methods.
The certificates are released at the end of the course if the candidate has to prove his identity and has attained the objectives of the course without external aid. Some udacity's courses are fee.
The Udacity’s courses are all self-paced they are organized primarily according modules. This modules have textual and video contributions, but also offer activities and short interviews with professors. In some cases they are provided English subtitles for the videos to which you can also adjust the speed of the speech.
Also the courses offer by Udacity have a showcase for ads, forum, wiki and community. For most of the courses are provided for the quiz intermediate, generally relating to a group of modules at a time, and a final test of the course. Unlike other platforms, on Udacity, quizzes can be repeated without this affecting the final grade.
%%%%
A year ago, Udacity has expanded its offering proposals Nanodegree the program, a kind of diploma for those who want to improve and further develop their skills. Among the courses offered there are front-end developer, data analyst, iOS developer, programming introduction
the success and the opportunity offered by the platform has obviously attracted the attention of Google which recently announced that it has launched in collaboration with the startup courses Tech Entrepreneur Nanodegree and Android Development

For Udacity, each student learns by following online classes that last from 6 to 9 months. The cost is \$ 200 a month and who finishes on time receives 50\% of the tuition fees back
%%%%



\subsection{Udemy}
\label{subsec:udemy}
\begin{figure}[htb] %  figure placement: here, top, bottom
 \centering
 \includegraphics[width=0.5\linewidth]{images/chapter1/udemy.jpg}\hfill
 \caption[Udemy logo]{Udemy logo}
 \label{fig:fourV}
\end{figure}

Udemy is platform  or marketplace for online learning that allow anyone to publish video online courses. There is a version of the site in Italian but the courses are in English and in other languages such as French, Spanish and Chinese.
There are free and paid courses in many different areas: excel, photoshop, facebook for companies, mathematical, but also cooking, music, sports, arts, languages. It 'an incredibly flexible platform especially for teachers, which automatically populate their video courses, determine their cost and the “positioning”.
 At the end of all courses released a certificate. Udemy provides a platform for experts of any kind to create courses which can be offered to the public, either at no charge or for a tuition fee, but in some cases is possible to see some of the first lessons for free.
The courses Udemy are all self-paced. They are organized in modules but they are not released weekly. Most of the video is not provided with subtitles. They are, moreover, all the most common tools of other platforms such as forums, communities, wikis and chat rooms, however there is a great and original function for taking notes during lessons.
Udemy has made a special effort to attract corporate trainers seeking to create coursework for employees of their company.[Carr, David F. Udemy Comes To Corporate Training Information Week. April 16, 2013]