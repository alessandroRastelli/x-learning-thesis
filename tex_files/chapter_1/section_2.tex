\section{MOOC classification}
\label{sec:mooc_classification}

In few years many MOOC services are developed, each of these has the particular features, policies, different services and among these i have selected those which can be considered the most important or however of the most relevant and to which I was inspired.

\subsection{Coursera}
\label{subsec:coursera}

\begin{figure}[htb] %  figure placement: here, top, bottom
 \centering
 \includegraphics[width=1.0\linewidth]{images/chapter1/coursera.jpg}\hfill
 \caption[Coursera logo]{Coursera logo}
 \label{fig:fourV}
\end{figure}

Coursera is a platform MOOC Established by Professors of Computer Science Andrew Ng and Daphne Koller Stanford University. This platform have more than 150 International  University Partners of Which anche the University of Rome La Sapienza and the Bocconi University of Milan, to allow the platform to offer courses in many different languages Among Which including Italian. All courses are accessible Coursera For free.
For those interested in some courses are given the opportunity to obtain, at the end of course, a "Verified certificate" a certificate issued by the university providing the course. The certificates are paid and have a variable cost.
Even Coursera courses generally begin at a precise date and have an average duration of a few weeks. They are divided into modules which are publishe online weekly. Forms can be made either by video contributions from both textual contributions. In the videos. Furthermore, the speed of the speech, can be adjusted and in some cases they are provided with subtitles in many different languages.
For most of the courses are provided for the quiz intermediate, generally relating to a group of modules at a time, and a final test of the course.
Each course has a discussion forums (divided into sections), a showcase for the publication of the ads, a community that helps you get to know other students of the course and wiki, useful as external support to complete course units. Unlike other platforms MOOC presented on this page, Coursera provides for the acceptance of an honor code before you can start your course.

\subsection{Udacity}
\label{subsec:udacity}
\begin{figure}[htb] %  figure placement: here, top, bottom
 \centering
 \includegraphics[width=1.0\linewidth]{images/chapter1/udacity.png}\hfill
 \caption[Udacity logo]{Udacity logo}
 \label{fig:fourV}
\end{figure}

Udacity was born from an experiment at Stanford University. The portal offers scientific, of design and business courses, divided by the various levels. The courses are provided to non-traditional teaching methods.
The certificates are released at the end of the course if the candidate has to prove his identity and has attained the objectives of the course without external aid. Some udacity's courses are fee.
The Udacity’s courses are all self-paced they are organized primarily according modules. This modules have textual and video contributions, but also offer activities and short interviews with professors. In some cases they are provided English subtitles for the videos to which you can also adjust the speed of the speech.
Also the courses offer by Udacity have a showcase for ads, forum, wiki and community. For most of the courses are provided for the quiz intermediate, generally relating to a group of modules at a time, and a final test of the course. Unlike other platforms, on Udacity, quizzes can be repeated without this affecting the final grade.

\subsection{Udemy}
\label{subsec:udemy}
\begin{figure}[htb] %  figure placement: here, top, bottom
 \centering
 \includegraphics[width=1.0\linewidth]{images/chapter1/udemy.jpg}\hfill
 \caption[Udemy logo]{Udemy logo}
 \label{fig:fourV}
\end{figure}

Udemy is platform  or marketplace for online learning that allow anyone to publish video online courses. There is a version of the site in Italian but the courses are in English and in other languages such as French, Spanish and Chinese. At the end of all courses released a certificate. Udemy provides a platform for experts of any kind to create courses which can be offered to the public, either at no charge or for a tuition fee, but in some cases is possible to see some of the first lessons for free.
The courses Udemy are all self-paced. They are organized in modules but they are not released weekly. Most of the video is not provided with subtitles. They are, moreover, all the most common tools of other platforms such as forums, communities, wikis and chat rooms, however there is a great and original function for taking notes during lessons.
Udemy has made a special effort to attract corporate trainers seeking to create coursework for employees of their company.[Carr, David F. Udemy Comes To Corporate Training Information Week. April 16, 2013]
