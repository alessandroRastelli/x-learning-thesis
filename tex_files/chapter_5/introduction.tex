This chapter analyzes the components that have been developed to manage video with the support of Amazon AWS service.
Each service presented in the second chapter will be detailed and contextualized through the description and snippets. 
Il primo capitolo è un panoramica sulla gestione di un corso e dei web component che vengono utilizzati.
Nel secondo e terzo capitolo verrà illustrato nel dettaglio il funzionamento del web component responsabile dell'upload e transcodifica dei contenuti video.
Nel quarto capitolo vedremo cosa si cela dietro al componente per la visualizzazione dei video.


% The first section provides a description of the various services used in the thesis project and will be explained by following a top-down model, starting from an overview and then adding implementation details.
% Therefore in the first section then we would have a very high-level description of the video management.
% In subsequent chapters it will be dealt with in detail the various services necessary for managing and streaming video.
The last section will be treated instead to use WebRTC and above peerjs created for the realization of the video-conference, and use in my thesis project for the realization of the webinar and then realtime lessons.