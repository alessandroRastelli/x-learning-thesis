\newpage
\section{Real time video}
\label{sec:RealTimeVideo}

%%%%
X-Learning in addition to video streaming are given the opportunity to the teacher to schedule webinars.
The webinar is a live event, allowing more people to connect live to participate in an interactive lesson, a training course, a workshop or conference. Just like in real classroom we find ourselves at the scheduled time and everyone has the opportunity to attend and speak at the event to ask questions and share ideas.
To achieve this feature has been adopted WebRTC useful in all contexts of video conference.
%%%%

%For the creation of server side webinar it was used PeerJs that allows the establishment and management of video conferences in a more simple and cleaner than WebRTC.

Client side instead was create two distinct a web component for the teacher and one for the student wishing to connect to the webinar.

\subsubsection{Webinar: server side}
To support the implementation of X-Learning I used JavaScript library PeerJS.
PeerJS uses implementing WebRTC Browser API to provide comprehensive, configurable and easy to use that allow the creation of the peer-to-peer. 


% With nothing more than an ID, a peer can create a P2P connection with a remote peer.Then every peer is Identified using nothing more than an ID. A string That the peer itself can choose, or have a server generated one.

PeerJS includes within it a module called PeerServer which it can be installed on its own server and is used to provide peer id which they will use to establish the P2P connection. The official repository on GitHub even has a one-click button to deploy an instance of PeerJS Server to Heroku.

In my case, i just want to create an instance of ExpressPeerServer in my Node.js application, and serve it at “/peerjs”

\begin{lstlisting}[language=javascript]
// peerjs server
  var options = {
      debug: true
  }
  app.use('/peerjs', ExpressPeerServer(server, options));

  server.listen(9000);
\end{lstlisting}

In una comunicazione P2p ogni peer deve essere identificato in maniera univoca. Quindi lato sever ogni peer che si connette a uno specifico webinar riceverà un ID univoco. 

These IDs can be generated from Peer Server for each peer automatically, or we can pick one for every peer while instantiating Peer objects.

% Once PeerJS Server up and running, let's implement the client side of the webinar. Each peer must be uniquely identified by an ID. These IDs can be generated from Peer Server for each peer automatically, or we can pick one for every peer while instantiating Peer objects.


\begin{lstlisting}[language=javascript]
var peer = this.peer = new Peer( { host: 'localhost', port: 9000, path: '/peerjs'});
\end{lstlisting}

As mentioned I want each peer is uniquely identified for this as we can see in the code above I have intentionally omitted the id and so PeerServer manage directly the choice and to ensure uniqueness.
The other options included in the creation of peer contain information such as host, port, path.


Once known ids of all peers can establish the connection through the method
peer.connect (destId) where destId in my case is the id of the teacher that is on hold.

\subsubsection{Webinar: client side}

Client side there is a distinction between teacher and student. Specifically, I created two Web Component:
\begin{enumerate}

\item The Web component in admin side can be invoked through the following snippet.
The use of such component performs the following operations:
\begin{itemize}
\item requires an id to the server listens
\item persists id webinar into model Webinar throght loopback(Mongodb)
\item sets up the call
\item waits
\end{itemize}

\begin{lstlisting}[language=html]
 <deck-p2p-admin data="{{data.webinar}}" on-peer="on_peer"></deck-p2p-admin>
\end{lstlisting}

\item The web component in client side instead allows a user enrolled in a course and interested in the webinar to be able to access the content.
The use of this component performs the following operations:
\begin{itemize}
\item performs a GET to get the id of communication
\item requires an id to the server PeerJs
\item establishes a connection with the call of the teacher
\item shows the video content in real time
\end{itemize}

\end{enumerate}
\begin{lstlisting}[language=html]
  <deck-p2p-client class="video" data="{{data}}"></deck-p2p-client>
\end{lstlisting}


