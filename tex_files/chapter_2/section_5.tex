\section{Amazon SNS and SQS}
\label{sec:Amazon SNS and SQS}


Amazon Simple Notification Service (Amazon SNS) is a distributed publish-subscribe system that has the task of managing the sending and receiving of messages.
Publishers communicate asynchronously with subscribers by producing and sending a message to a topic, which is a logical access point and communication channel. Subscribers (i.e., web servers, email addresses, Amazon SQS queues, AWS Lambda functions) consume or receive the message or notification over one of the supported protocols (i.e., Amazon SQS, HTTP/S, email, SMS, Lambda) when they are subscribed to the topic.\cite{aws_sns}


You don't always have to unite SNS and SQS, but in my platform both are being used.

By uniting SNS with SQS, you can receive messages at your own pace. It allows clients to be offline, tolerant to network and host failures. You also achieve guaranteed delivery.

Amazon SQS is a distributed queue system that enables web service applications to quickly and reliably queue messages that one component in the application generates to be consumed by another component. A queue is a temporary repository for messages that are awaiting processing.

Messages can contain up to 256 KB of text in any format. Any component can later retrieve the messages programmatically using the Amazon SQS API.

Amazon SQS supports multiple readers and writers interacting with the same queue and a single queue can be used simultaneously by many distributed application components.

In my platform the order is not important. In fact, when Elastic Transcoder job is completed a message was produced and sent.
At this time the subscriber SQS receives the notification and inserts a message in the queue. X-learning server-side read and processed each SQS message in order to publish the course.