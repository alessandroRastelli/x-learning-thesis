\section{Amazon AWS}
\label{sec:Amazon AWS}

Amazon Web Services (AWS) is a cloud-computing platform offered by Amazon.com and can be seen as a collection of remote computing services.
In 2006 Amazon understands that its infrastructure and its web services, which currently are used only for herself, could be shared with other companies who need a reliable infrastructure for their own business.

The services of Amazon Web Services are divided and distributed in places completely autonomous and independent of each other, located in different places of the world. These services operate from 11 geographical regions (Regions), in each region there are a number of infrastructure called availability zones.
The choice of the geographical area of Amazon AWS is very important, because it influences the latency and so the response time, normally you should choose the closest geographical area, the costs for the same service are different from Brazil, USA, Europe, etc.
AWS is located in 11 geographical ``regions'': US East (Northern Virginia), where the majority of AWS servers are based,\cite{aws_stats1} US West (northern California), US West (Oregon), Brazil (São Paulo), Europe (Ireland and Germany), Southeast Asia (Singapore), East Asia (Tokyo and Beijing) and Australia (Sydney).

Each Region has multiple ``Availability Zones'', which are distinct data centers providing AWS services. Availability Zones are isolated from each other to prevent outages from spreading between Zones. Several services operate across Availability Zones (e.g., S3, DynamoDB) while others can be configured to replicate across Zones to spread demand and avoid downtime from failures. Amazon web services hold 1.79\% market share. As of December 2014, Amazon Web Services operated an estimated 1.4 Million servers across 28 availability zones \cite{aws_stats2}.
