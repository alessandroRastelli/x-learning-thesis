\section{WebRTC: Overview}
\label{sec:WebRTC: Overview}


http://www.toptal.com/webrtc/taming-webrtc-with-peerjs

http://www.html5rocks.com/en/tutorials/webrtc/basics/


%%%%
X-Learning in addition to video streaming are given the opportunity to the teacher to schedule webinars.
The webinar is a live event, allowing more people to connect live to participate in an interactive lesson, a training course, a workshop or conference. Just like in real classroom we find ourselves at the scheduled time and everyone has the opportunity to attend and speak at the event to ask questions and share ideas.
To achieve this feature has been adopted WebRTC useful in all contexts of video conference.
%%%%

WebRTC born with the intention to offer Web Developers One Tool to manage the interchange of data stream between two devices direct connection ( peer-to -Peer ).
WebRTC (Web Real-Time Communication) is an API definition drafted by the World Wide Web Consortium (W3C) that supports browser-to-browser applications for voice calling, video chat, and P2P file sharing without the need of either internal or external plugins.[ How WebRTC Is Revolutionizing Telephony. Blogs.trilogy-lte.com (2014-02-21). Retrieved on 2014-04-11.]

Since it was first introduced by Google in May 2011, WebRTC has been used in many modern web applications. Being a core feature of many modern web browsers, web applications can seamlessly take advantage of this technology to deliver improved user experience in many ways. Video streaming or conferencing applications that don’t require bloated browser plugins, and can take advantage of peer-to-peer (P2P) networks (while not transmitting every bit of data through some server) is only a part of all the amazing things that can be achieved with WebRTC[http://www.toptal.com/webrtc/taming-webrtc-with-peerjs]
WebRTC is used in various apps like WhatsApp, Facebook Messenger, appear.in and platforms such as TokBox[http://www.html5rocks.com/en/tutorials/webrtc/basics/]
