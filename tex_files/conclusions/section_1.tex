\section{Work performed}
\label{sec:work_performed}
 
In conclusione come si è visto nei capitoli precedenti lo studio e l'unione di tecnologie diverse a reso possibile la realizzazione di un sistema di e-learning.
In X-Learning il caso d'uso analizzato, sono stati sfruttati i benifici lato client derivanti da Polymer e Web Components, Server-side benefits are represented by the easy way of API creation through model definitions.
Code reusability and access to reusable code are, clearly, one of the most important aspects of the project that lead to the implementation of a set of elements.
Gli elementi principali creati durante il progetto di tesi sono certamente quelli utili alla gestione e streaming dei video su cui si basa l'intera piattaforma.
Sono quindi stati creati diversi elementi stand alone che incapsulano l'intera complessità e che possono essere riutilizzati in diversi contesti e progetti.
Inoltre si è affrontato anche uno studio approfondito e utlizzato ampiamente le API che permettono l'interazione con i vari servizi di Amazon AWS quali multiPartUpload, Elastic Transcoder, SQN, Cloufront.
Infine è stata dedicata l'ultima parte del progetto nello studio di WebRTC e nella realizzazione attraverso PeerJs di componenti stand alone che permettono di creare delle lezioni peer to peer.