In conclusion we laid the foundations for an ambitious project that allows a user to create their own E-Learning platform.

In X-Learning, the use case analyzed, the benefits of Polymer Web Components on client side have been used, while Server-side benefits are represented by the easy way of API creation through model definitions.
Code reusability and access to reusable code are, clearly, one of the most important aspects of the project that lead to the implementation of a set of elements.

The main components created during the thesis project are certainly those useful to the management and streaming video content, which the entire platform is based on.

There were then created several stand-alone elements that encapsulate the entire complexity and they can be reused in different contexts and projects.

The interaction between the various services such multiPartUpload, Elastic Transcoder, SQN, Cloufront was made possible thanks to the use of different APIs made available by the latter.

The in-depth study of the various Amazon AWS services has been extremely important. This study has made the creation of these components possible.

In addition, the study of the standard w3c WebRTC and PeerJS library has made possible the creation of stand-alone components which allow the peer to peer communication used in X-learning to expand course offerings with the addition of Webinars.