\section{Work performed}
\label{sec:work_performed}
 
In conclusione sono state gettate le basi per un progetto ambizioso che consente a un utente di realizzare la propria piattaforma di e-learning.

In X-Learning il caso d'uso analizzato, sono stati sfruttati i benifici lato client derivanti da Polymer e Web Components, Server-side benefits are represented by the easy way of API creation through model definitions.
Code reusability and access to reusable code are, clearly, one of the most important aspects of the project that lead to the implementation of a set of elements.

I componenti principali creati durante il progetto di tesi sono certamente quelli utili alla gestione e allo streaming dei contenuti video su cui si basa l'intera piattaforma.

Sono quindi stati creati diversi elementi stand alone che incapsulano l'intera complessità e che possono essere riutilizzati in diversi contesti e progetti.

L'interazione tra i vari servizi quali multiPartUpload, Elastic Transcoder, SQN, Cloufront è stata possibile grazie all'utilizzo delle diverse API messe a disposizione.
E' risultato quindi di fondamentale importanza lo studio approfondito dei vari servizi Amazon AWS che ha reso possibile la realizzione di tali componenti.
% Attraverso l'utilizzo delle diverse API  e quindi 

Inoltre è stato necessario approfondire anche lo studio dello standard w3c WebRTC e della libreria PeerJS che ha reso possibile la realizzazione di componenti stand alone che permettono la comunicazione peer to peer, sfruttata in X-Learning per ampliare l'offerta dei corsi con l'aggiunta dei Webinar.