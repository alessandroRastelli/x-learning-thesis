In conclusione è stato realizzato un servizio di e-learning e sono state gettate le basi per un progetto ambizioso che permette a un utente col minimo sforzo di realizzare una propria piattaforma di e-learning o a una azienda di creare una sezione Accademy.
Per raggiungere il traguardo è stato di fondamentale importanza uno studio approfondito dei vari servizi Amazon AWS che ha reso possibile la realizzione di tali componenti tramite l'utilizzo delle API e quindi l'interazione con i vari servizi quali multiPartUpload, Elastic Transcoder, SQN, Cloufront.
In X-Learning il caso d'uso analizzato, sono stati sfruttati i benifici lato client derivanti da Polymer e Web Components, Server-side benefits are represented by the easy way of API creation through model definitions.
Code reusability and access to reusable code are, clearly, one of the most important aspects of the project that lead to the implementation of a set of elements.
Gli elementi principali creati durante il progetto di tesi sono certamente quelli utili alla gestione e streaming dei video su cui si basa l'intera piattaforma.
Sono quindi stati creati diversi elementi stand alone che incapsulano l'intera complessità e che possono essere riutilizzati in diversi contesti e progetti.
Infine è stata dedicata l'ultima parte del progetto nello studio di WebRTC e nella realizzazione attraverso PeerJs di componenti stand alone che permettono di creare delle lezioni peer to peer.