\section{Future developments}
\label{sec:future_developments}
 
Al momento X-Learning rappresenta il core di un progetto ambizioso che vuole dare la possibilità all'utente, o all'azienda di creare la propria piattaforma di e-learning.

Quindi tra gli sviluppi futuri sicuramente il principale è il deploy, ovvero incapsulare tutto la piattaforma e creare una macchina Docker con all'interno tutte le tecnologie necessarie in modo tale che chiunque si iscriva al servizio avrà la propria piattaforma su una macchina dedicata già preconfigurata.

Di seguito sono riportati i principali task in programma che in ottica futura renderebbero il progetto da me intrapreso completo.

\begin{itemize}
  \item Integrare la piattaforma, con servizi offerti da piattaforme concorrenti come ad esempio forum e chat tra utenti iscritti agli stessi corsi.

  \item Effettuare il porting da loopback a forester, ovvero un framework sviluppato all'interno del CVDLAB, che guarda al futuro sfruttando tecnologie moderne e non ancora particolarmente diffuse come koajs e le async/await ma che diventeranno presto essenziali 

  \item Creare nuovi temi concedendo massima autonomia agli utenti di scegliere il tema più consono al stile della loro piattaforma.

  \item Realizzare attraverso PHONEGAB delle app mobile ibride rendendo disponibile i servizi anche su mobile attraverso gli store di google play e appleStore.

  \item Infine creare un app nativa desktop attraverso l'utilizzo di ELECTRON.

\end{itemize}