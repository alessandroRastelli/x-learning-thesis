\section{Future developments}
\label{sec:future_developments}

Al momento il caso d'uso X-learning è in uno stato molto avanzato ma ancora non definitivo del progetto che si vuole portare a termine.
Per questo sono già stati identificati alcuni punti da cui ripartire per poter rendere definitivamente disponibile un servizio completo e allo stesso tempo pronto ad offrire la possibilità di creare una propria piattaforma.

Tra gli sviluppi futuri sicuramente il principale è il deploy, ovvero incapsulare tutta la piattaforma creando uno strato superiore che per ogni utente istanzi una macchina Docker dedicata e già preconfigurata con all'interno tutte le tecnologie necessarie.

Di seguito sono riportati i principali obiettivi che in ottica futura renderebbero il progetto intrapeso definitivmente completo.

\begin{itemize}
  \item Integrare la piattaforma, con servizi offerti dalle piu comuni piattaforme MOOC come ad esempio forum e chat tra utenti iscritti agli stessi corsi.

  \item Effettuare il porting da loopback a forester, ovvero un framework sviluppato all'interno del CVDLAB, che guarda al futuro sfruttando tecnologie moderne e non ancora particolarmente diffuse come koajs e le async/await ma che diventeranno ben presto essenziali. 

  \item Creare nuovi temi concedendo massima autonomia agli utenti di scegliere il tema più consono al stile della loro piattaforma.

  \item Use Adobe PhoneGap framework which is an open source distribution of Cordova that allow to quickly make hybrid applications.

  % Infine creare un app nativa desktop attraverso l'utilizzo di ELECTRON.
  \item Finally, through the Electron use, which is open source, maintained by GitHub and an active community, allow to build cross platform desktop apps with web technologies.\cite{electron}

\end{itemize}