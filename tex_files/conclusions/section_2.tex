\section{Future developments}
\label{sec:future_developments}

Currently, the use case of X-learning is in a very advanced state but still hasn't arrived at the final desired phase.
For this reason, some tasks have already been identified in order to make a service that is complete and permanently available which is at the same time ready to offer the possibility of creating a personal platform.

Among the future developments, the main one is definitely the deploy, or the encapsulation of the entire platform creating a top layer that allows each user to instantiate a dedicated and preconfigured Docker machine with all the necessary technology within it.

Below are the main objectives that in future perspective would make the project definitely complete.



\begin{itemize}
  \item Integrating the platform with services offered by the most common MOOC platforms such as forums and chat between subscribers of the same courses.

  \item Porting to forester, which is a framework developed within the CVDLAB, which moves toward the future by using modern technology and still not particularly popular such as koajs and async / await but which will soon become essential.

  \item Creating new themes granting maximum autonomy to the users allowing them to choose the theme best suited to the style of their platform.

  \item The use Adobe PhoneGap framework which is an open source distribution of Cordova allows to quickly make hybrid applications.

  % Infine creare un app nativa desktop attraverso l'utilizzo di ELECTRON.
  \item Finally, through the Electron use, which is open source, maintained by GitHub and an active community, allows to build cross platform desktop apps with web technologies.\cite{electron}

\end{itemize}