\section{Amazon AWS}
\label{sec:Amazon AWS}

Amazon Web Services (AWS), a collection of remote computing services, also called web services, make up a cloud-computing platform offered by Amazon.com.[2] These services operate from 11 geographical regions across the world. The most central and well-known of these services arguably include Amazon Elastic Compute Cloud and Amazon S3. Amazon markets these products as a service to provide large computing-capacity more quickly and more cheaply than a client company building an actual physical server farm [4].
AWS is located in 11 geographical “regions”: US East (Northern Vir- ginia), where the majority of AWS servers are based,[19] US West (northern California), US West (Oregon), Brazil (São Paulo), Europe (Ireland and Germany), Southeast Asia (Singapore), East Asia (Tokyo and Beijing) and Australia (Sydney). There is also a “GovCloud”, based in the Northwestern United States, provided for U.S. government customers, complementing existing government agencies already using the US East Region.[4] Each Region is wholly contained within a single country and all of its data and services stay within the designated Region.[citation  needed]
Each Region has multiple “Availability Zones”, which are distinct data centers providing AWS services. Availability Zones are isolated from each other to prevent outages from spreading between Zones. Several services operate across Availability Zones (e.g., S3, DynamoDB) while others can be configured to replicate across Zones to spread demand and avoid downtime from failures. Amazon web services hold 1.79\% market share. As of December 2014, Amazon Web Services operated an estimated 1.4 Million servers across 28 availability zones [18].
